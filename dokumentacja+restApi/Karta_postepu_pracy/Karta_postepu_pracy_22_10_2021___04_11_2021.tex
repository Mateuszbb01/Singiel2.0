\documentclass[12pt,a4paper]{article}
\usepackage[polish]{babel}
\usepackage[T1]{fontenc}
\usepackage[utf8]{inputenc}



\addtolength{\hoffset}{-1.5cm}
\addtolength{\marginparwidth}{-1.5cm}
\addtolength{\textwidth}{3cm}
\addtolength{\voffset}{-1cm}
\addtolength{\textheight}{2.5cm}
\setlength{\topmargin}{0cm}
\setlength{\headheight}{0cm}

\begin{document}

\title{Karta postępu pracy}
\date{22.10.2021 - 04.11.2021}

\maketitle


\section*{Tytuł aplikacji}
SINGIEL 2.0
\section*{Nazwa robocza aplikacji}
Aplikacja randkowa z komunikatorem
\section*{Cel}
Stworzenie aplikacji do łączenia się w pary użytkowników z możliwością czatowania.
\section*{Zespół}
Bartłomiej Tokarczyk  (project menager), \\
Wojciech Sułowski, \\
Mateusz Sromek, \\
Maciej Więcławek \\
\newpage

\section{Zaplanowane do zrealizowania zadania:}

\begin{itemize}
    \item [--] Implementacja preferencji użytkownika w RestApi \textbf{(Mateusz Sromek pomoc - Maciej Więcławek)}

    \item [--] Zakładka edycji profilu (aplikacja mobilna) – layout \textbf{(Bartłomiej Tokarczyk)}

    \item [--] Wdrożenie obsługi profilów użytkownika w RestApi \textbf{(Mateusz Sromek pomoc - Wojciech Sułowski)}

    \item [--] Integracja api z funkcją logowania oraz rejestracji \textbf{(Wojciech Sułowski)}
    \item [--] Przygotowanie layoutu pod RestApi – logowanie/rejestracja w aplikacji \textbf{(Bartłomiej Tokarczyk)}
    \item [--] Testy RestApi przy użyciu aplikacji PostMan, wskazanie błędów przy konfiguracji RestApi \textbf{(Bartłomiej Tokarczyk)}
    \item [--] Diagram przypadków użycia oraz diagram przepływu
    \item [--] Konfiguracja bazy danych wraz z obsługą POST/GET/PUT/DELETE \textbf{(Maciej Więcławek pomoc – Mateusz Sromek)}
    \item [--] Instalacja Docker, konfiguracja kontenera z Laravel Api + MySql \textbf{(Mateusz Sromek pomoc – Maciej Więcławek)}

\end{itemize}

\section{Zrealizowane założenia:}
\begin{itemize}
    \item [--] Implementacja preferencji użytkownika w RestApi 
    \item [--] Zakładka edycji profilu (aplikacja mobilna) – layout    
    \item [--] Wdrożenie obsługi profilów użytkownika w RestApi aplikacji
    \item [--] Integracja api z funkcją logowania oraz rejestracji
    \item [--] Przygotowanie layoutu pod RestApi – logowanie/rejestracja w aplikacji
    \item [--] Testy RestApi przy użyciu aplikacji PostMan, wskazanie błędów przy konfiguracji RestApi
    \item [--]Konfiguracja bazy danych wraz z obsługą POST/GET/PUT/DELETE
 
\end{itemize}

\section{Poświęcony czas na realizację projektu:}
\begin{itemize}
    \item [--] Wojciech Sułowski  - 18h
    \item [--] Mateusz Sromek  - 24h
    \item [--] Maciej Więcławek - 32h
    \item [--] Bartłomiej Tokarczyk - 20h
\end{itemize}
Zaplanowany czas na realizację wyznaczonych zadań: \textbf{30h}

\section{ Napotkane problemy podczas realizacji zadań}
\begin{itemize}
    \item [--] Błędne wyznaczenie ilości czasu potrzebnego do realizacji poszczególnych  zadań
    \item [--] Problem z czasem oczekiwania na odpowiedź od Docker do aplikacji mobilnej (wydłużony czas odpowiedzi serwera dla aplikacji, co skutkowało brakiem możliwości rejestracji oraz logowania) \\
\textbf{Rozwiązanie problemu:} Zmiana podsystemu WSL na V-Hyper
 \item [--] Problem z wyświetlaniem editboxów po kompilacji programu (ucinany tekst przy wpisywaniu loginu oraz hasła)\\
 \textbf{Rozwiązanie problemu:} Implementacja biblioteki android material design oraz zmiana   typu editboxów.


  \item [--] Niezgodność plików lokalnych w Android Studio co powodowało nie kompilowanie się aplikacji \\
\textbf{Rozwiązanie problemu:} Ręczna podmiana plików, zignorowanie plików lokalnych przy kolejnych zapisach aplikacji
   \item [--] Problem z nie generowaniem się tokenu przy rejestracji użytkownika w api  co uniemożliwiało dodawanie preferencji użytkownika \\
\textbf{Rozwiązanie problemu:} Generowanie tokenu przy rejestracji użytkownika w metodzie POST (Laravel)

    \end{itemize}
    \\
    Zadania przeniesione z sprintu 1 do sprintu 2:
\begin{itemize}
 \item [--] Rejestracja preferencji użytkownika - plec, zainteresowania, zdjecie, opis.
 \item [--] Zakładka edycja profilu (aplikacji mobilna)
  \item [--] Backend do edycji profilu usera 
\end{itemize}
    
\section{Zadania zaplanowane na najbliższy sprint 05.10.2021 - 18.10.2021: }
\begin{itemize}
   \item [--] Rejestracja preferencji użytkownika - płeć, zainteresowania, zdjecie, opis. \textbf{(Maciej Więcławek) (wykonane 08.10.2021)}
 \item [--]Zakładka edycja profilu (aplikacji mobilna) - layout \textbf{(Bartłomiej Tokarczyk) (wykonane 09.10.2021)}

  \item [--]Backend do edycji profilu usera\textbf{ (Mateusz Sromek) (wykonane 08.10.2021)}

   \item [--]Obsługa zakładki przeglądania użytkowników (swipe right/left) z poziomu RestApi. \textbf{(Mateusz Sromek)}

    \item [--]Layout do strony głównej aplikacji; zakładka przeglądania użytkowników w android studio \textbf{(Maciej Więcławek)}

     \item [--]Backend do zakładki przeglądania użytkowników. Obsługa połączenia pomiędzy zakładką a RestApi.\textbf{ (Maciej Więcławek)}

      \item [--]Implementacja technologii socketIO do obsługi czatowania - po stronie RestApi. \textbf{(Mateusz Sromek)}
      
      \item [--]Przygotowanie bazy danych, utworzenie routingów/kontrolerów. Dodanie funkcjonalności szyfrowania wiadomości tekstowych.\textbf{ (Bartłomiej Tokarczyk) }

      \item [--]Przygotowanie layoutu pod czat użytkowników\textbf{ (Wojciech Sułowski) }

      \item [--]Backend do wyświetlania sparowanych użytkowników (zakładka czat) \textbf{(Wojciech Sułowski)}
      
 \item [--]Backend do konwersacji ze sparowanym użytkownikiem. \textbf{(Bartłomiej Tokarczyk)}

\end{itemize}

Planowany, łączny czas realizacji zadań:\textbf{ 83h}


\end{document}