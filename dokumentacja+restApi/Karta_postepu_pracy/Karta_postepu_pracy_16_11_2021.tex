\documentclass[12pt,a4paper]{article}
\usepackage[polish]{babel}
\usepackage[T1]{fontenc}
\usepackage[utf8]{inputenc}



\addtolength{\hoffset}{-1.5cm}
\addtolength{\marginparwidth}{-1.5cm}
\addtolength{\textwidth}{3cm}
\addtolength{\voffset}{-1cm}
\addtolength{\textheight}{2.5cm}
\setlength{\topmargin}{0cm}
\setlength{\headheight}{0cm}

\begin{document}

\title{Karta postępu pracy}
\date{5.11.2021 - 18.12.2021}

\maketitle


\section*{Tytuł aplikacji}
SINGIEL 2.0
\section*{Nazwa robocza aplikacji}
Aplikacja randkowa z komunikatorem
\section*{Cel}
Stworzenie aplikacji do łączenia się w pary użytkowników z możliwością czatowania.
\section*{Zespół}
Bartłomiej Tokarczyk  (project menager), \\
Wojciech Sułowski, \\
Mateusz Sromek, \\
Maciej Więcławek \\
\newpage

\section{Zaplanowane do zrealizowania zadania:}

\begin{itemize}
    \item [--] Rejestracja preferencji użytkownika - płeć, zainteresowania, zdjecie, opis. \textbf{(Maciej Więcławek)}

    \item [--] Zakładka edycja profilu (aplikacji mobilna) - layout  \textbf{(Bartłomiej Tokarczyk)}

    \item [--] Backend do edycji profilu usera \textbf{(Mateusz Sromek)}

    \item [--] Obsługa zakładki przeglądania użytkowników (swipe right/left) z poziomu RestApi.  \textbf{(Mateusz Sromek)}
    \item [--] Layout do strony głównej aplikacji; zakładka przeglądania użytkowników w android studio \textbf{(Maciej Więcławek)}
    \item [--] Backend do zakładki przeglądania użytkowników. Obsługa połączenia pomiędzy zakładką a RestApi. \textbf{(Maciej Więcławek)}
    \item [--] Implementacja technologii socketIO do obsługi czatowania - po stronie RestApi. {(Mateusz Sromek)}
    \item [--] Przygotowanie bazy danych, utworzenie routingów/kontrolerów. Dodanie funkcjonalności szyfrowania wiadomości tekstowych. \textbf{(Bartłomiej Tokarczyk)}
    \item [--] Przygotowanie layoutu pod czat użytkowników \textbf{(Wojciech Sułowski) }
    \item [--] Backend do wyświetlania sparowanych użytkowników (zakładka czat) \textbf{(Wojciech Sułowski)}
    \item [--] Backend do konwersacji ze sparowanym użytkownikiem. \textbf{(Bartłomiej Tokarczyk)}
\end{itemize}

\section{Zrealizowane założenia:}
\begin{itemize}
    \item [--] Obsługa zakładki przeglądania użytkowników (swipe right/left) z poziomu RestApi. (Mateusz Sromek)
    \item [--] Zakładka edycja profilu (aplikacji mobilna) - layout (Bartłomiej Tokarczyk) 
    \item [--] Layout do strony głównej aplikacji; zakładka przeglądania użytkowników w android studio (Maciej Więcławek)
    \item [--] Backend do zakładki przeglądania użytkowników. Obsługa połączenia pomiędzy zakładką a RestApi. (Maciej Więcławek)
    \item [--] Rejestracja preferencji użytkownika - płeć, zainteresowania, zdjecie, opis. (Maciej Więcławek)

\end{itemize}

\section{Poświęcony czas na realizację projektu:}
\begin{itemize}
    \item [--] Wojciech Sułowski - 8h (brak realizacji zaplanowanego zadania do 14.11) Realizacja 15.11-16.11 (planowany czas 14h)
    \item [--] Mateusz Sromek  - 20h
    \item [--] Maciej Więcławek - 35h
    \item [--] Bartłomiej Tokarczyk - 7h
\end{itemize}
Zaplanowany czas na realizację wyznaczonych zadań: \textbf{30h}
\section{Zadania zaplanowane na najbliższy sprint 19.11.2021 - 02.12.2021: }
\begin{itemize}
   \item [--] Obsługa przesyłania plików po stronie serwera
 

 \item [--]Implementacja rozmów wideo w restapi (webRtc - firebase)

  \item [--]Layout do rozmów wideo

   \item [--]Layout do przesyłania plików
    \item [--]Layout do przesyłania kontaktów

     \item [--] Backend do rozmów wideo
      \item [--]Backend do przesyłania plików
      \item [--]Backend do przesyłania kontaktów NFC


\end{itemize}

Planowany, łączny czas realizacji zadań:\textbf{ 86h}

\section{ Napotkane problemy podczas realizacji zadań}
\begin{itemize}

    \item [--]  Przekazywanie danych pomiędzy aktywnościami związanymi z rejestracją preferencji
 \\
\textbf{Rozwiązanie problemu:} Zapisywanie danych z poszczególnych formularzy poprzez SharedPreferences, a następnie wysyłanie skumulowanych danych do RestApi.
 \item [--] Błąd przy wysyłaniu danych do RestApi spowodowany niepoprawnym wypełnieniem poszczególnych pól formularza rejestracji preferencji.
\\
 \textbf{Rozwiązanie problemu:} Implementacja walidacji przy użyciu biblioteki material design oraz walidacja danych przed wysłaniem do RestApi.


  \item [--] Błędne wyświetlanie zdjęć w zakładce “swipe” \\
\textbf{Rozwiązanie problemu:} Dodanie funkcji odpowiedzialnej za kompresję zdjęć.

   \item [--] Problem z przekazywaniem danych o aktualnie swipe’owanym użytkowniku do RestApi \\
\textbf{Rozwiązanie problemu:} Przekazanie danych o użytkowniku do SharedPreferences w momencie wyświetlania kafelka, a następnie w zależności od wykonanej akcji (swipe right, swipe left) wysłanie danych do RestApi.



   \item [--] Kafelki nie wyświetlają się po zalogowaniu użytkownika \\
\textbf{Rozwiązanie problemu:} Wywoływanie funkcji odpowiedzialnej za tworzenie tablicy z danymi za każdym razem gdy uruchamia się aktywność “HomeActivity”.

    \end{itemize}
    \\



\end{document}