\documentclass[12pt,a4paper]{article}
\usepackage[polish]{babel}
\usepackage[T1]{fontenc}
\usepackage[utf8]{inputenc}



\addtolength{\hoffset}{-1.5cm}
\addtolength{\marginparwidth}{-1.5cm}
\addtolength{\textwidth}{3cm}
\addtolength{\voffset}{-1cm}
\addtolength{\textheight}{2.5cm}
\setlength{\topmargin}{0cm}
\setlength{\headheight}{0cm}

\begin{document}

\title{Karta postępu pracy}
\date{19.10.2021}

\maketitle


\section*{Tytuł aplikacji}
SINGIEL 2.0
\section*{Nazwa robocza aplikacji}
Aplikacja randkowa z komunikatorem
\section*{Cel}
Stworzenie aplikacji do łączenia się w pary użytkowników z możliwością czatowania.
\section*{Zespół}
Bartłomiej Tokarczyk  (project menager), \\
Wojciech Sułowski, \\
Mateusz Sromek, \\
Maciej Więcławek \\
\newpage

\section{Zaplanowane do zrealizowania założenia:}
\textbf{Realizacja poszczególnych punktów dokumentacji projektu tj:}
\begin{itemize}
    \item [--] Tytuł aplikacji
    \item [--] Nazwa robocza aplikacji
    \item [--] Cel aplikacji
    \item [--] Analiza wymagań aplikacji
    \item [--] Deasemblacja aplikacji
    \item [--] Wymagania funkcjonalne oraz niefunkcjonalne
    \item [--] Diagram przypadków użycia oraz diagram przepływu
    \item [--] Dobór technologii
    \item [--] Scenariusze przypadków użycia
    \item [--] Estymacja czasowa
    \item [--] określenie MVP rejestracji
    \item [--] Roboczy layout
\end{itemize}

\section{Zrealizowane założenia:}
\begin{itemize}
    \item [--] Określenie tytułu, celu oraz nazwy roboczej aplikacji
    \item [--] Przeprowadzenie analizy wymagań oraz deasemblacji projektu aplikacji
    \item [--] Określenie wymagań funkcjonalnych oraz niefunkcjonalnych aplikacji
    \item [--] Stworzenie diagramu przypadków użycia oraz diagramu przepływu aplikacji
    \item [--] Dobranie odpowiednich technologii do realizacji projektu
    \item [--] Wykonanie scenariuszów przypadku użycia
    \item [--] Analiza oraz rozpisanie estymacji czasowej projektu aplikacji
    \item [--] Określenie MVP (minimum viable product) projektu aplikacji
    \item [--] Stworzenie wersji roboczej Layout’u
    \item [--] Aktualizacja bibliografii
\end{itemize}

\section{Poświęcony czas na realizację projektu:}
\begin{itemize}
    \item [--] Wojciech Sułowski  - 5 spotkań po 3h = 15h
    \item [--] Mateusz Sromek  - 5 spotkań po 3h = 15h
    \item [--] Maciej Więcławek - 5 spotkań po 3h = 15h
    \item [--] Bartłomiej Tokarczyk -  5 spotkań po 3h = 15h
\end{itemize}
\textbf{Łącznie: }60h \\
\textbf{Zaplanowany czas na realizację dokumentacji projektu: }80h

\section{ Napotkane problemy podczas realizacji dokumentacji}
\hspace{5mm} Jedynym problemem w realizacji dotychczasowych zagadnień tj. dokumentacji projektu było dokładne, teoretyczne zrozumienie poszczególnych elementów dokumentacji przed ich realizacją. Podczas pracy zespołowej wszelkie problemy zostały rozwiązane, a praca nad większą, zaplanowaną na ten czas częścią dokumentacji została zrealizowana pomyślnie.

\section{Zaplanowane dalsze założenia do realizacji:}
\begin{itemize}
   \item [--] Rozpoczęcie pracy nad implementacją funkcjonalności oraz Layout’u projektu aplikacji
\end{itemize}
\textbf{ Zaplanowany czas realizacji nowych założeń:  }250h
 
\section{Zaplanowane założenia do wykonania, do następnych zajęć:}
\begin{itemize}
   \item [--] Implementacja layotu w wersji roboczej
   \item [--] Konfiguracja bazy danych
\end{itemize}



\end{document}